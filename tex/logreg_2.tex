\documentclass{ltjsarticle}
\usepackage{luatexja} 
\usepackage{bm}
\title{ロジスティック回帰入門(2)}
\author{FooQoo}
\begin{document}
\maketitle
\section{概要}
調査結果を解釈する指標として、「リスク比」と「オッズ比」がある。
本稿では、これらの指標についての理解、その違い、有意差の算出方法を紹介する。

\section{分割表}
リスク比やオッズ比、そして有意差を算出するためには、分割表が必要になる。
表\ref{tbl:sample}は10人の患者について調べたものである。
このデータに対して、不整脈が喫煙者と非喫煙者を比較した時、両者に差があるか確認する。

\begin{table}[b]
    \caption{データ例}
    \label{tbl:sample}
    \centering
    \begin{tabular}{r|r|r} \hline
        \multicolumn{1}{l}{患者No.} & \multicolumn{1}{|l|}{喫煙有無} & \multicolumn{1}{l}{不整脈有無} \\ \hline \hline
        1 & 0 & 0 \\ \hline
    \end{tabular}
\end{table}

データを読み取ると、喫煙者において不整脈の割合は$60\%$で、非喫煙者における不整脈のある人の割合は$20\%$であることがわかる。
この結果を表にしたものを分割表(contingency table)と呼び、表\ref{tbl:c}として表す。

\begin{table}[b]
    \caption{分割表}
    \label{tbl:c}
    \centering
    \begin{tabular}{l|r|r|r|r} \hline 
        \multicolumn{1}{l}{喫煙有無} & \multicolumn{2}{|l|}{不整脈有無} & \multicolumn{1}{|l|}{横計} & \multicolumn{1}{l}{割合}\\ \cline{2-3}
        & ある & ない & & \\ \hline \hline
        喫煙 & 3 & 1 & 5 & $60\%$ \\
        非喫煙 & 1 & 4 & 5 & $20\%$ \\ \hline 
    \end{tabular}
\end{table}


\begin{thebibliography}{1}
    \bibitem{istat} 株式会社アイスタット https://istat.co.jp/sk\_commentary/risk\_odds
\end{thebibliography}
\end{document}
