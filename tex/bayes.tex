\documentclass{ltjsarticle}
\usepackage{luatexja} 
\usepackage{bm}
\title{ベイズ統計学入門}
\author{FooQoo}
\begin{document}
\date{2019/1/14}
\maketitle
\section{確率の復習}
\subsection{確率の定義}
事象$A$の起こる確率$p$を次のように一般化する。
ただし起こりうる全ての事象は等確率で発生する。
\begin{equation}
    p=\frac{事象Aの起こる場合の数}{起こりうる全ての場合の数}
\end{equation}
起こりうる全ての事象の集合$U$を標本空間と呼ぶ。
$U$の個数に対する、事象$A$が発生する要素の個数で割った値を「数学的確率」とする。
一方で、何回も試行を繰り返した時、得られる事象$A$の割合を「統計的確率」と呼ぶ。

\subsection{同時確率}
事象$A$と$B$を考える。
$A$と$B$が同時に起こる事象を$A \cap B$と表す。
この確率を、$P(A \cap B)$と表し、同時確率と呼ぶ。

\subsection{条件付き確率}
ある事象$A$が起こった条件のもとで事象$B$の起こる確率を、条件付き確率と呼ぶ。
条件付き確率は、$P(B|A)$で表し、以下の式で計算される。
\begin{equation}
    P(B|A)=\frac{P(A \cap B)}{P(A)}
\end{equation}

確率の定義のみ用いた証明は以下。
\begin{eqnarray}
    P(B|A)=\frac{n_{A \cap B}}{n_A}, & & P(A \cap B)=\frac{n_{A \cap B}}{n_U}, P(A)=\frac{n_A}{n_U}より \nonumber \\
    P(B|A)=& &\frac{n_{A \cap B}}{n_A} \\
    =& &\frac{n_U}{n_A} \frac{n_{A \cap B}}{n_U} \\
    =& &\frac{1}{\frac{n_A}{n_U}}\frac{n_{A \cap B}}{n_U} \\
    =& &\frac{P(A \cap B)}{P(A)}
\end{eqnarray}

\subsection{確率の乗法定理}
同時確率と条件付き確率を整理すると以下の式が導出される。
\begin{equation}
    P(A \cap B)=P(A) P(B|A)
\end{equation}

\end{document}
